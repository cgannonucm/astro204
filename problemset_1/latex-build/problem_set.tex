\documentclass{article}
% Template borrowed from https://latex-tutorial.com/tutorials/first-document/
\title{Problem Set 1}
\date{October 10th 2024}
\author{Charles Gannon}
\begin{document}
  \pagenumbering{gobble}
  \maketitle
  \pagenumbering{arabic}
  (a) The gravitational acceleration enacted on our test object is
  \begin{equation}
      \vec{a} = -\frac{G M}{r^2}\hat{r}, 
  \end{equation}
  where G is the gravitational constant, M is the mass of the deflector, $r$ is the separation between the particles and $\hat{r}$ is unit vector pointing towards the point mass from the deflector.
  Using the impulse approximation 
  \begin{equation}\label{impulse}
    \Delta v = a \Delta t, 
  \end{equation}
  where $\Delta v$ is the change in velocity, $a$ is the acceleration at closest approach and $\Delta t$ is the relevant timescale near the deflector.
  Assuming that minimal deflection, the distance between the deflector and the point mass will be $\sim b$, and assuming we change the velocity of order $v_{rel} $  we can plug into eq. \ref{impulse} to get 
  \begin{equation}
      v_{rel} = \frac{G M}{b^2} \Delta t ,
  \end{equation}
  Since for our situation the relavent timescale is 
  \begin{equation}
      \Delta t \sim \frac{2b}{v_{rel}},
  \end{equation}
  we have 
  \begin{equation}
      b = 2 \frac{G M}{v_{rel}^2},
  \end{equation}
  Finally, we can substitute 
  \begin{equation}
      v_{esc} = \left( \frac{2 G M}{R} \right) ^ 2
  \end{equation}
  to get 
  \begin{equation}
      b = R \left( \frac{v_{esc}}{v_{rel}} \right)^2
  \end{equation} 

  (b) Conservation of energy gives 
  \begin{equation}
      (1/2) m v_{rel}^2 = (1/2) m v_{surf}^2 - \frac{G M m}{R},
  \end{equation}
  Where $v_{surf}$ is the velocity of the particle at the surface.
  Conservation of angular momentum gives 
  \begin{equation}
    v_{rel} b = v_{surf} R
  \end{equation}
  Combining these two equations we have 
  \begin{equation}
      b = \sqrt{1 + \frac{2 G M}{R v_rel^2}}R = \sqrt{1 + \left( \frac{v_{esc}}{ v_{rel}} \right)^2 R}
  \end{equation}
  (c) See jupyter notebook
  
\end{document}
