% Created 2024-11-24 Sun 20:22
% Intended LaTeX compiler: pdflatex
\documentclass[11pt]{article}
\usepackage[utf8]{inputenc}
\usepackage[T1]{fontenc}
\usepackage{graphicx}
\usepackage{longtable}
\usepackage{wrapfig}
\usepackage{rotating}
\usepackage[normalem]{ulem}
\usepackage{amsmath}
\usepackage{amssymb}
\usepackage{capt-of}
\usepackage{hyperref}
\usepackage{enumitem}
\usepackage{breqn}
\author{Charles Gannon}
\date{\today}
\title{Astro 204 Problemset 7}
\hypersetup{
 pdfauthor={Charles Gannon},
 pdftitle={Astro 204 Problemset 7},
 pdfkeywords={},
 pdfsubject={},
 pdfcreator={Emacs 29.4 (Org mode 9.8)}, 
 pdflang={English}}
\begin{document}

\maketitle
\tableofcontents

\section{Effective Viscosity}
\label{sec:org67aae07}
(a) The viscosity for a fluid is
\begin{equation}
 \nu = v_{cs} \lambda_{\text{mfp}}.
\end{equation}
We can extend this to our effective fluid, assigning an effective speed of sound \(v_{\text{eddy}}\) and mean free path \(L_{\text{eddy}}\).
The effective speed of sound cannot exceed the actual speed of sound in the fluid, therefore
\begin{equation}
 v_{eddy} < c_s.
\end{equation}
Additionally, the eddy must be physically contained within the disk, so the length scale associated with each eddy, \(L_{\text{eddy}}\) must be less than the scale height of the disk
\begin{equation}
 L_{\text{eddy}} < H.
\end{equation}
Combining these two observations, we can write the effective viscosity as
\begin{equation}
 \nu = \alpha c_s H,
\end{equation}
with \(\alpha < 1\) parameterizing our ignorance of the underlying physics.
\newline
(b) The energy per wavenumber is \(k E_k (k) \sim k^{-2/3}\).
This energy should be proportional to the kinetic energy so \(k^{-2/3} \sim \frac{1}{2} m v^2\), so \(v \sim k^{-1/3}\).
The effective viscosity is proportional to the velocity and the length of the scale so
\begin{equation}
 \nu = v_{\text{eddy}} L_{\text{eddy}} \sim k^{-1/3} / k = k^{-4/3} \sim l^{4/3}
\end{equation}
so larger length scales contribute more to the velocity.
\newline
(c) Plugging in gives a viscosity
\begin{equation}
 \nu = \alpha c_s H = (10^{-2}) (2 \cdot 10^{6} cm/s) (8 \cdot 10^{11} m) \sim 10^{15} cm^2 / s.
\end{equation}
At \(r=1AU\) This corresponds to a timescale of
\begin{equation}
 t = r^2 / \nu = \left(1.5 \cdot 10^{13} cm \right)^2 / \left( 10^{15} cm^2/s  \right) \sim 10^{11} s \sim 10^{3} y
\end{equation}
which is much more reasonable.
\section{Temperature of an accretion disk}
\label{sec:org181564b}
(a) A generic form of the energy equation for fluids is
\begin{equation}
 \rho T \frac{ds}{dt} = \psi - \lambda,
\end{equation}
integrating over z gives
\begin{equation}
 \Sigma T \frac{ds}{ds} = 2 H \psi - 2 H \lambda = Q_+ - Q_-.
\end{equation}
We can convert the total derivative to a partial derivative
\begin{equation}
 \frac{d}{dt} = \frac{\partial }{\partial t} + \vec{v} \cdot \vec{\nabla}.
\end{equation}
Noting that in steady state \(\frac{\partial }{\partial t} = 0\) and that under the assumption of disk symmetry we have \(\vec{v} \cdot \vec{\nabla} = v_r \frac{d}{dr}\) we have
\begin{equation}
 \Sigma v_r T \frac{ds}{dr} = Q_+ - Q_- .
\end{equation}
\newline
(b) Since we can expect orbits to be circular because of damping, the total gravitational energy of the system is proportional to the mass squared
\begin{equation}
 E \sim -\frac{G M^2}{r}.
\end{equation}
Taking the derivative gives
\begin{equation}
 \dot{E} \sim -\frac{G M \dot{M}}{r}.
\end{equation}
The area of the orbit is proportional to \(r^2\), therefore
\begin{equation}
 Q_+ \propto r^2,
\end{equation}
therefore
\begin{equation}
 Q_+ \sim \frac{G M \dot{M}}{r^3}.
\end{equation}
Plugging in the expression from Kepler's 3rd law \(GM = \Omega^2 r^3\) I get
\begin{equation}
 Q_+ \sim \Omega^2 M
\end{equation}
\newline
(d) From thermodynamics \(Tds = d \epsilon - \frac{P}{\rho^2}d\rho\), where \(\epsilon  = \frac{1}{\gamma - 1} c_s^2\).
Converting to 2d, and plugging in a polytropic equation of state gives
\begin{equation}
 Tds = \frac{1}{\gamma - 1} c_s^2 - c_s^2 \frac{d \Sigma}{\Sigma}
\end{equation}
Therefore we have
\begin{equation}
 \Sigma v_r T \frac{ds}{dr} = \frac{1}{\gamma - 1} \Sigma \frac{d c_s^2}{dr} - c_s^2 v_r \frac{d \Sigma}{dr}.
\end{equation}
Both terms are order of magnitude \(\frac{\Sigma v_r c_s^2}{r}\), and multiplying by r on the top and bottom we get and noting \(r \Sigma v_r \sim \dot{M}\)
\begin{equation}
 \frac{r \Sigma v_r c_s^2}{r^2} \sim \frac{\dot{M} c_s^2}{r^2}.
\end{equation}
From part b, we have  \(Q_+ \sim \frac{\dot{M}v_k^2}{r^2}\) .
For a thin disk, \(c_s << v_k\), therefore
\begin{equation}
 \Sigma v_r T \frac{ds}{dr} << Q_+.
\end{equation}
(e) The total amount of radiative cooling is equal to the total lumnosity output.
Therefore, the total luminosity is
\begin{equation}
 L = \int_{R_*}^\infty 2 \pi r Q_- dr \sim \frac{G M \dot{M}}{2 R_*}.
\end{equation}
\newline
(e) We can relate the total temperature of the disk to the radiative cooling
\begin{equation}
 Q_- = 2 \sigma_{sb} T^4.
\end{equation}
Since we have \(Q_- \sim Q_+\), the temperature as a function of radius is
\begin{equation}
 T^4 = \frac{3 G M \dot{M}}{8 \pi \sigma_{sb} r^3} = \frac{3 \Omega^2 \dot{M}}{8 \pi \sigma_{sb}}.
\end{equation}
Plugging in \(\dot{M} = 10^{-8} M_\odot / yr\) at 1 AU (the earth's orbit, so
\(\Omega = 1yr^{-1}\)), I get \(T = 52.6 K\) so heating wins.
(f) If the photons went straight, it would take them \(t = H/c\) to diffuse out of the disk.
However, they diffuse so the time to diffuse is \(\left(H / \Lambda_{\text{mfp}} \right)^2 / c\) to diffuse out of the disk.
The optical depth of the disk is
\begin{equation}
 \tau = n \sigma H = H / \lambda_{\text{mfp}},
\end{equation}
therefore, the Luminosity is a factor of \(\tau\) larger than the Luminosity at the surface.
Since \(L \sim \sigma_{\text{sb}} T^4\), the relationship between the temperature at the surface and the midplane is
\begin{equation}
 T_m \sim T_s \tau^{1/4}
\end{equation}
\section{Time evolution of an accretion disk}
\label{sec:orgc5ded80}
(a) The mass equations is
\begin{equation}
 \frac{\partial \rho}{\partial t} + \vec{\nabla} \cdot (\rho \vec{v}) = 0,
\end{equation}
integrating over z and imposing symmetry gives
\begin{equation}
 \frac{\partial \Sigma}{\partial t} + \frac{1}{r} (r \Sigma v_r) = 0
\end{equation}
The momentum equation is
\begin{equation}
 \rho \frac{d \vec{v}}{dt} = - \vec{\nabla} P + \rho \vec{f},
\end{equation}
we can rewrite the left hand side and drop terms, and  dropping terms with time dependence
\begin{equation}
 \rho \frac{d \vec{v}}{dt} = \rho (\vec{v} \cdot \vec{\nabla}) \vec{v}
\end{equation}
Plugging in the viscous force per mass, Keppler's law, dropping all other terms on the right and integrating over mass gives
\begin{equation}
 \Sigma v_r \frac{\partial }{\partial r} (r^2 \Omega) = \frac{1}{r^2}\frac{\partial }{\partial r} \left( \Sigma \nu r^3 \frac{\partial \Omega}{\partial r}  \right) .
\end{equation}
Keppler's 3rd law gives into the momentum equation and solving for \(v_r\) gives
\begin{equation}
 v_r = -3 \Sigma^{-1} r^{-1/2} \frac{\partial }{\partial r}(\Sigma \nu r^{1/2})
\end{equation}
which we can plug into the mass equation to get
\begin{equation}
 \frac{\partial \Sigma}{\partial t} = \frac{3}{r} \frac{\partial }{\partial r} \left[ r^{1/2} \frac{\partial }{\partial r} \left( \Sigma \nu r^{1/2} \right)  \right]
\end{equation}
\newline
(b) Plugging in \(\gamma = 1\), \(R = \frac{r}{r_0}\), i\(T = \frac{t}{t_{visc}} + 1\) and \(C = \Sigma_0 e\) we get
\begin{equation}
 \Sigma = C R^{-1} T^{-3/2} e^{-R / T}.
\end{equation}
Plugging in our substitution variables \(T\) and \(R\) into the main equation we get
\begin{equation}
 \frac{\partial \Sigma}{\partial T} = \frac{1}{R} \frac{\partial }{\partial R} \left[ R^{1/2} \frac{\partial }{\partial R} \left( \Sigma \nu R^{1/2} \right)  \right].
\end{equation}
Now, we can plug everything in.
The left hand side of the equation becomes
\begin{equation}
 C R^{-1} \left( -\frac{3}{2} T^{-5/2} e^{-R/T} + T^{-3/2} e^{-R/T} R T^{-2}  \right) =
\end{equation}
\begin{equation}
 C R^{-1} T^{-3/2} e^{-R / T} \left( -\frac{3}{2} T^{-1} + RT^{-2}  \right) =
\end{equation}
\begin{equation}
\Sigma \left( -\frac{3}{2} T^{-1} + RT^{-2}  \right)
\end{equation}
The right hand side becomes
\begin{equation}
\frac{1}{R} \frac{\partial }{\partial R} \left[ R^{1/2} \frac{\partial }{\partial R} \left( C T^{-3/2} R^{1/2} e^{-R/T} \nu R^{1/2} \right)  \right] =
\end{equation}
\begin{equation}
 C T^{-3/2} \frac{1}{R} \frac{\partial }{\partial R} \left[  \frac{1}{2} e^{-R/T} - R T^{-1} e^{-R/T}  \right] =
\end{equation}
\begin{equation}
 C T^{-3/2} \frac{1}{R} \frac{\partial }{\partial R} \left[ e^{-R/T}\left( \frac{1}{2} - R T^{-1} \right)  \right] =
\end{equation}
\begin{equation}
 C T^{-3/2} R^{-1} \left[ -T^{-1} e^{-R/T} \left( \frac{1}{2} - RT^{-1} \right) - T^{-1}e^{-R/T}  \right] =
\end{equation}
\begin{equation}
 C T^{-3/2} R^{-1} e^{-R/T} \left( -\frac{3}{2} T^{-1} + RT^{-2} \right) =
\end{equation}
\begin{equation}
 \Sigma \left( -\frac{3}{2} T^{-1} + RT^{-2} \right) .
\end{equation}
Both sides match which means our solution is valid.
\newline
(c) See jupyter notebook
\end{document}
