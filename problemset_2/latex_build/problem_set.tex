\documentclass{article}
% Template borrowed from https://latex-tutorial.com/tutorials/first-document/
\title{Problem Set 2}
\date{October 10th 2024}
\author{Charles Gannon}
\begin{document}
  \pagenumbering{gobble}
  \maketitle
  \pagenumbering{arabic}
  \begin{enumerate}
    \item Hydrostatic atmospheres
    \begin{enumerate}
        \item 
            To calculate the scale height of an atmosphere with pressure
            \begin{equation}
                \frac{dP}{dz} =  P_0 e^{-z / H},
            \end{equation}
            we first calculate
            \begin{equation}
                \frac{dP}{dz} =  - \frac{P_0}{H} \cdot e^{-z / H}
            \end{equation}
            then take the ratio
            \begin{equation}
                P(z) / \left| dP(z)/dz \right| = H
            \end{equation}
        \item
            The hydrostatic balance equation is
            \begin{equation}\label{eq_hydro_balance}
                \frac{1}{\rho} \frac{dP}{dr} = -G \frac{M}{r^2}.
            \end{equation}
            The density, $\rho$ can be expressed in terms of pressure
            \begin{equation}
                P = \rho c_s^2,
            \end{equation}
            assuming an ideal gas.
            Furthermore, for the isothermal case, $c_s$ is constant with respect to radius.
            Recasting eq. \ref{eq_hydro_balance} in a more easily integrable form gives 
            \begin{equation}
                \frac{1}{\rho} \frac{dP}{dr} = -\frac{G M}{c_s^2} \frac{1}{r^2} = \frac{1}{H_0} \left( \frac{R_0}{r} \right)^2
            \end{equation}
            which can be integrated to give
            \begin{equation}
                P = A e^{\frac{R_0^2}{H_0} \frac{1}{r}}.
            \end{equation}
            Imposing the condition $P(R) = P_0$ gives 
            \begin{equation}\label{eq_pshere}
                P = P_0 \exp \left[ {\left ( \frac{R}{H_0} \right ) \left( \frac{R}{r} - 1\right)} \right] 
            \end{equation}
        \item
            Rewriting the exponent in eq. \ref{eq_pshere} in terms of z = r - R gives
            \begin{equation}
                - \frac{R}{H_0} \left [ z/R (1 + z/R)^{-1} \right ], 
            \end{equation}
            which we can Taylor expand for $z << R$
            \begin{equation} 
                z/R (1 + z/R)^{-1} \sim z/R (1 - z/R) \sim z/R,
            \end{equation}
            which can be substituted into eq. \ref{eq_pshere} to give  
            \begin{equation}
                P = P_0 \exp \left[ - z / H_0  \right] \approx P_0 exp \left[ -z / R \right],
            \end{equation}
            which is identical to the plane parallel case.        
        \item 
            \begin{equation}
                \lim_{t \rightarrow \infty} P = P_0 \exp \left[ - R / H_0 \right] \neq 0
            \end{equation} 
    \end{enumerate}
    \item Optical depth and scale lengths (see jupyter notebook)
    \item The Lyman-$\alpha$ forest.
        For a cloud to a part of the forest, the optical depth, $\tau$ must be $\tau \approx 1$.
        The optical depth of a Lyman $\alpha$ cloud is 
        \begin{equation}
            \tau = N_{HI} \sigma_0.
        \end{equation}
        Therefore, for the cloud to be in the forest
        \begin{equation}
            N_{HI} = 1 / \sigma_0, 
        \end{equation}

        From the notes, the cross-section of Lyman-$\alpha$ absorption is 
        \begin{equation}
            \sigma_0 = \frac{1}{8 \pi} \frac{g_1}{g_2} \frac{A_{21}}{\Delta v} \Lambda^2
        \end{equation}
        for an order of magnitude estimate assume states have equal weighting so $g_1 \approx g_2$, therefore
        \begin{equation}
            \sigma_0 \approx \frac{A_{21}}{\Delta v} \Lambda^2
        \end{equation}
        The speed of sound is $c_s \approx \sqrt{\frac{k T}{\nu}} \approx 10^6 cm / s$
        Plugging in $\Lambda \approx 1.2 \cdot 10^{-5} cm$, $\delta_v \approx c_s / \Lambda_0 \approx 10^{11} 1/s$ the crosssection is $\sigma_0 \approx 10^{-12} cm^{2}$. 
        This gives a required column density of $N_{HI} \approx 10^{12}$ cm$^{-2}$.
    \item Rate coefficients for chemical reactions. 
        To estimate the boor radius, take the angular momentum as the quantized quantity 
        \begin{equation}
            \hbar = m_e a_0 v_c
        \end{equation}
        where the orbital velocity, $v_c$ can be estimated from uniform circular motion,
        \begin{equation}
            k \frac{e^2}{a_0^2} = m_e v_c^2 / a_0.
        \end{equation}
        Solving for $a_0$ gives 
        \begin{equation}
            a = \frac{\hbar^2}{k e^2 m_e} \approx \sim 1\cdot 10^{-8} [cm]
        \end{equation}
        The number of collisions occurring per time is
        \begin{equation}
            n = \sigma v n_1 n_2,
        \end{equation}
        therefore 
        \begin{equation}
            k = v \sigma,
        \end{equation} 
        where v is the thermal velocity
        \begin{equation}
            1/2 m v^2 = 3/2 k_b T,
        \end{equation}
        and $\sigma$ is the cross-section of hydrogen
        \begin{equation}
            \sigma \sim \pi a_0^2,
        \end{equation}
        which can approximated using our estimate for the Bohr radius, which gives an approximate cross-section of $\sigma \approx 3 \cdot 10^{-16}$ cm$^2$. 
        The thermal velocity for the $10^4$ K gas is $v \approx  1 \cdot 10^-6$ cm/s which gives a final estimate for $k$ as $k \approx 10^{-10} cm^3 / s$ which is close to the $q_{21}$ value provided.  
  \end{enumerate} 
\end{document}
