% Created 2024-11-02 Sat 01:14
% Intended LaTeX compiler: pdflatex
\documentclass[11pt]{article}
\usepackage[utf8]{inputenc}
\usepackage[T1]{fontenc}
\usepackage{graphicx}
\usepackage{longtable}
\usepackage{wrapfig}
\usepackage{rotating}
\usepackage[normalem]{ulem}
\usepackage{amsmath}
\usepackage{amssymb}
\usepackage{capt-of}
\usepackage{hyperref}
\usepackage{enumitem}
\usepackage{breqn}
\author{Charles Gannon}
\date{\today}
\title{Astro 204 Problem Set 4}
\hypersetup{
 pdfauthor={Charles Gannon},
 pdftitle={Astro 204 Problem Set 4},
 pdfkeywords={},
 pdfsubject={},
 pdfcreator={Emacs 29.4 (Org mode 9.8)}, 
 pdflang={English}}
\begin{document}

\maketitle
\tableofcontents

\section{Numerical Time-Evolution to a Parker Wind}
\label{sec:orgea8080f}
(a) The mass conservation equation
\begin{equation}
 \frac{\partial \rho}{\partial t} + \vec{\nabla} \cdot (\rho \vec{v}) = 0,
\end{equation}
simplifies to
\begin{equation}
 \frac{\partial \rho}{\partial t} + \frac{1}{r^2}\left(  r^2 \rho v \right )  = 0,
\end{equation}
under the assumption of spherical symmetry, which can be expanded to
\begin{equation}
 \frac{\partial \rho}{\partial t} + \frac{2}{r} \rho v + v \frac{\partial \rho}{\partial r} + \partial \frac{\partial v}{\partial r}  = 0,
\end{equation}
using the chain rule.
The momentum conservation equation is
\begin{equation}
 \rho \frac{d \vec{v}}{dt} = - \vec{\nabla} P + \rho \vec{f}
\end{equation}
which can be expanded by applying the chain rule
\begin{equation}
 \frac{d \vec{v}}{d t} = \left( \frac{\partial v}{\partial t} + v \frac{\partial  v}{\partial r} \right ) \hat{r} ,
\end{equation}
Additionally upon substituting the isothermal equation of state \(P = c_s^2 \rho\) the gradient of pressure becomes
\begin{equation}
 \vec{\nabla} P = c_s^2 \frac{\partial \rho}{\partial r} \hat{r}.
\end{equation}
Combining everything, we get
\begin{equation}
 \rho \frac{\partial v}{\partial t} + \rho v \frac{\partial v}{\partial r} = -c_s^2 \frac{\partial \rho}{\partial r} - \rho \frac{G M}{r^2}
\end{equation}
(b) Upon substitution of \(x = ln r\) the equation (1) becomes
\begin{equation}
 \frac{\partial \rho}{\partial t} + e^{-2x} \frac{d }{dx}(e^{2x}\rho v) = 0
\end{equation}
and equation 2 becomes
\begin{equation}
 e^{x} \left ( e^{x} \rho \frac{\partial v}{\partial t} + \rho v \frac{\partial v}{\partial x} + c_s^2 \frac{\partial \rho}{\partial x} \right ) =  - GM \rho.
\end{equation}
(c) See jupyter notebook
\section{Wave Breaking in the Upper Atmosphere}
\label{sec:org2801f4b}
(a) For our, the mass conservation equation is
\begin{equation}
 \frac{\partial \rho}{\partial t} + \vec{\nabla} \cdot (\rho \vec{v}) = 0
\end{equation}
and momentum conservation is
\begin{equation}
 \rho \frac{d \vec{v}}{dt} = - \vec{\nabla} P + g \rho \hat{z},
\end{equation}
if we take the near surface limit for the gravitational force.
Now, we add perturbations \(P_1\), \(\rho_1\), \(v_1\) to a set of know solutions, \(P_0\), \(\rho_0\) and \(v_0\), like the derivation in class, I choose \(v_0 = 0\).
First, I calculate the scale height of the atmosphere.
Under the assumption \(v_0 = 0\)
\begin{equation}
 \vec{\nabla} P_1 = \frac{d P_1}{d \rho _1} \vec{\nabla} \rho_1 =g \rho_1 \hat{z},
\end{equation}
which upon substituting \(a_0^2 = \frac{d P}{d \rho}\) and expanding \(\vec{\nabla} \rho\) the equation becomes
\begin{equation}
  a_0^2 \frac{\partial \rho_1}{\partial z} \hat{z} = g \rho_1 \hat{z},
\end{equation}
which gives a scale height
\begin{equation}
  H = \rho / \left | \frac{\partial \rho_1}{\partial z} \right | = \frac{a_0^2}{g}.
\end{equation}
Applying perturbation theory to the first equation, gives the same result as for the case without considering gravity (the solution we derived in class)
\begin{equation}
 \frac{\partial \rho_1}{\partial t} + \rho_0 \vec{ \nabla } \cdot \vec{v}_1 = 0.
\end{equation}
Applying perturbation theory to the second equation we and applying similar arguments as were applied in class get
\begin{equation}
 \rho_0 \frac{d \vec{v}}{dt} = -a_0^2 \vec{\nabla} \rho_1 - g \rho_1 \hat{z}
\end{equation}
Combining the energy and momentum equation into the wave equation, we get
\begin{equation}
 \frac{\partial^2 \rho_1}{\partial t^2}
 - a_0^2 \nabla^2 \rho_1
 - \frac{\partial \rho_1}{\partial z}
 = 0.
\end{equation}
Assuming the solution
\begin{equation}
 \rho = \delta \rho e^{i (\vec{k} \cdot \vec{x} - \omega t)}
\end{equation}
and plugging we get
\begin{equation}
 - \omega^2 + a_0^2 k^2 - i g k_z = 0
\end{equation}
which for a wave traveling in the z direction becomes
\begin{equation}
 \omega^2 = a_0^2 \left (k^2 - i \frac{k}{H} \right )
\end{equation}
after substituting \(H = a_0^2 / g\)
(b) Rearranging for \(k\), we have
\begin{equation}
 k^2 - \frac{i k}{H} - \frac{\omega^2}{a_0^2} = 0,
\end{equation}
solving using the quadratic equation we get
\begin{equation}
 k = \frac{i}{2H} \pm \sqrt{-\frac{1}{(2H)^2} + \frac{\omega^2}{a_0^2}}.
\end{equation}
Plugging into our assumed solution, we get
\begin{equation}
 \rho_1 = \delta \rho e^{-z/(2H)} e^{\text{Imaginary Part}}
\end{equation}
the fractional amplitude is
\begin{equation}
 \frac{\Delta \rho}{\rho_0} = \frac{\delta}{\rho_0} e^{-z / (2H)}
\end{equation}
however the background density decreases with height as \(\rho \propto e^{-z/H}\) so the fractional amplitude is proportional to
\begin{equation}
 \frac{\Delta \rho}{\rho_0} \propto \frac{\Delta \rho}{\rho_0} e^{-z / (2H)} e^{z/H} = e^{z/(2H)}
\end{equation}
so the fractional density amplitude increases with height.
\end{document}
