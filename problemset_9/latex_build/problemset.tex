% Created 2024-12-09 Mon 14:21
% Intended LaTeX compiler: pdflatex
\documentclass[11pt]{article}
\usepackage[utf8]{inputenc}
\usepackage[T1]{fontenc}
\usepackage{graphicx}
\usepackage{longtable}
\usepackage{wrapfig}
\usepackage{rotating}
\usepackage[normalem]{ulem}
\usepackage{amsmath}
\usepackage{amssymb}
\usepackage{capt-of}
\usepackage{hyperref}
\usepackage{enumitem}
\usepackage{breqn}
\author{Charles Gannon}
\date{\today}
\title{Astro 204 Problemset 9}
\hypersetup{
 pdfauthor={Charles Gannon},
 pdftitle={Astro 204 Problemset 9},
 pdfkeywords={},
 pdfsubject={},
 pdfcreator={Emacs 29.4 (Org mode 9.8)}, 
 pdflang={English}}
\begin{document}

\maketitle
\tableofcontents

\section{The Dynamical Time}
\label{sec:orgf3da453}
(a) Using the form of Kepler's 3'rd law
\begin{equation}
 G M = \Omega^2 a^3,
\end{equation}
the period is equal to
\begin{equation}
 P = \frac{2 \pi}{\Omega} = 2 \pi \sqrt{\frac{a^3}{GM}}.
\end{equation}
The free fall time is just half the orbital time
\begin{equation}
 t_{ff} = \frac{P}{2} =  \pi \sqrt{ \frac{a^3}{GM} }.
\end{equation}

(b) For a circular orbit
\begin{equation}
 \frac{G M }{r^2} = \frac{v_{\text{circ}}^2}{r}
\end{equation}
therefore,
\begin{equation}
 v_{\text{circ}} = \sqrt{\frac{GM}{r}}
\end{equation}
Using Kepler's third law,
\begin{equation}
  v_{\text{circ}} = \Omega r
\end{equation}
or equivalently
\begin{equation}
 \Omega_{\text{circ}} = \frac{v_{\text{circ}}}{r},
\end{equation}
which gives an equivalent orbital period of
\begin{equation}
 P_{\text{circ}} = 2 \pi \sqrt{\frac{r^3}{G M}}.
\end{equation}


(c) See python file / plots directory. The orbital period is 1 year.



(d) See python file / plots directory. The orbital period is about 1/2 a year (looking at the plots).



(e) The Escape velocity can be quickly found by setting the kinetic energy equal to the gravitational potential energy
\begin{equation}
 \frac{1}{2} m v_{\text{esc}}^2 = \frac{G M m}{r},
\end{equation}
which gives an escape velocity
\begin{equation}
 v_{\text{esc}} = \sqrt{\frac{2 G M}{r}}.
\end{equation}
At constant velocity, the time it takes to travel distance r  is
\begin{equation}
 t_{\text{esc}} = \frac{r}{v_{\text{esc}}} = \sqrt{\frac{r^3}{2 G M}}.
\end{equation}


(f) See python file / plots directory. The orbit gets larger with increasing velocity, up to about 4 times longer at \(v_0 = 1.4 v_{circ}\).


(g) This is the time scale gravitation works on.
The free-fall time, collapse time of a cloud, orbital time are all within a factor of unity.

(h) Substituting \(M = \frac{4}{3} \pi r^3\) into the free fall time I get
\begin{equation}
 t_{\text{collapse}} = \pi \sqrt{ \frac{3}{4 G \rho} } \sim \left(  G \rho \right)^{-1/2} .
\end{equation}
\section{Tidal Gravity}
\label{sec:org78c7df3}
The Swarzschild radius of black hole is
\begin{equation}
 r_s = \frac{2 G M_{bh}}{c^2}.
\end{equation}
Neglecting relativistic effects, and dropping order the Roche limit of the black hole is
\begin{equation}
 r_t \sim R_* \left( \frac{M_{bh}}{M_{*}}   \right)^{1/3},
\end{equation}
where \(M_{bh}\) is the mass of the black hole, \(M_*\) is the mass of the star and \(R_*\) is the radius of the star.L
If the Roche limit is greater than the Swarzschild radius the star will be visibly tidally disrupted.
On the other hand if \(r_t < r_s\) the star will be swallowed whole.
Dropping order of unity coefficients, for the star to be swallowed whole,
\begin{equation}
 M_{bh} > \left ( \frac{R_* c^2}{G M^{1/3}} \right )^{3/2}.
\end{equation}
For a stellar mass star (\(M_* = M_\odot\) and \(R_* = 1 R_\odot\)) I get that the black hole must have mass \(M_{bh} > 3 \cdot 10^8\) to swallow the star whole.
\end{document}
