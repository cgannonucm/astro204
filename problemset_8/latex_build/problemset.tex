% Created 2024-12-02 Mon 19:35
% Intended LaTeX compiler: pdflatex
\documentclass[11pt]{article}
\usepackage[utf8]{inputenc}
\usepackage[T1]{fontenc}
\usepackage{graphicx}
\usepackage{longtable}
\usepackage{wrapfig}
\usepackage{rotating}
\usepackage[normalem]{ulem}
\usepackage{amsmath}
\usepackage{amssymb}
\usepackage{capt-of}
\usepackage{hyperref}
\usepackage{enumitem}
\usepackage{breqn}
\author{Charles Gannon}
\date{\today}
\title{Astro 204 Problem Set 8}
\hypersetup{
 pdfauthor={Charles Gannon},
 pdftitle={Astro 204 Problem Set 8},
 pdfkeywords={},
 pdfsubject={},
 pdfcreator={Emacs 29.4 (Org mode 9.8)}, 
 pdflang={English}}
\begin{document}

\maketitle
\tableofcontents

\section{Turkey (or tofurkey)}
\label{sec:orgfce2d69}
Since heating a turkey is a diffusive process, the number of diffusive steps required to heat the turkey is
\begin{equation}
 n = \left( \frac{r}{\lambda}  \right)^2,
\end{equation}
where r is the radius of the turkey and \(\lambda\) is the mean free path between atoms in the turkey.
Each diffusive step takes time
\begin{equation}
 t_{\text{step}} = \frac{\lambda}{v_{th}}
\end{equation}
where \(v_{th}\) is the thermal velocity of phonons in the turkey.
Therefore, the time to cook the turkey is
\begin{equation}
 t = \frac{r^2}{\lambda v_{th}}.
\end{equation}
The thermal velocity of the phonons can be estimated using the spring model mentioned in the hint.
The effective spring constant
\begin{equation}
 k_{\text{spr}} = 2 \frac{U_{\text{spr}}}{\lambda^2}
\end{equation}
can be estimated by assuming the energy of the oscillator is \(U_{\text{spr}} \sim 1 ~\text{eV}\) (this is reasonable choice for the energy of a chemical system.)
The thermal velocity of the phonons can be estimated using the phonons can be estimated by using the spring model given in the hint.
The spring constant can be related to the energy of the oscillator
\begin{equation}
  k_{\text{sp}} = 2 \frac{U_{\text{sp}}}{\lambda^2},
\end{equation}
where \(U_{\text{sp}} \sim 1 ~ \text{eV}\) is a good order of magnitude estimate for a chemical system.
The thermal velocity
\begin{equation}
 v_{th} \sim \lambda \omega
\end{equation}
can then be estimated using the frequency of the oscillator
\begin{equation}
 \omega = \sqrt{\frac{k_{\text{sp}}}{m}}.
\end{equation}
Assuming the turkey is made almost entirely out of water \(m = 18 m_{p}\), and the lattice spacing is \(r_{\text{lattice}} \sim 2 \dot{A}\) and multiply by a factor of a few (3) to estimate the mean free path \(\lambda \sim 5 \dot{A}\).
I get an effective spring constant
\begin{equation}
  k_{\text{sp}} = 2 \frac{1.6 \cdot 10^{-12} ~ \text{erg}}{\left (5 \cdot 10^{-8} ~ \text{cm} \right )^2 } = 1280 ~ \text{erg} ~\text{cm}^{-2},
\end{equation}
and a thermal velocity
\begin{equation}
 v \sim \left( 5 \cdot 10^{-8} \text{cm} \right ) \sqrt{\frac{1280 ~ \text{erg} ~\text{cm}^{-2}}{18 \cdot 1.6 \cdot 10^{-24} \text{g}}} \sim 3 ~\text{km} ~\text{s}^{-1}.
\end{equation}
Plugging everything in, and assuming a turkey 10 cm in radius, I get a total time to cook the turkey
\begin{equation}
 t = \frac{\left( 10 \text{cm}  \right)^2 }{\left( 5 \cdot 10^{-8} \text{cm} \right)^2 \left( 3 \cdot 10^5 ~\text{cm} ~\text{s}^{-1}  \right)  } \sim 6 \cdot 10^3 \text{s} \sim 1.7 \text{hr}
\end{equation}
equal to about an hour and a half.
So pretty close, but maybe a bit on the short end, very good for an order of magnitude estimate!
\end{document}
